
Основным результатом выполнения проекта будет мобильное приложение, предназначенное для преобразования 2D изображений в 3D вид. Приложение будет распространяться с помощью его размещения в Google Play (для Android-устройств). Соответственно, в качестве основных потребителей создаваемой продукции следует рассматривать владельцев мобильных устройств, которые любят использовать свой телефон или планшет в качестве фотоаппарата. Более того, то подмножество этих пользователей, которые, помимо фотографирования, активно обрабатывают свои фото средствами мобильного устройства и активно делятся этими результатами с друзьями посредством соцсетей.

Поэтому главной целью текущей НИР является исследованию	преобразования двумерных изображений в трехмерные и разработке современного, функционального и удобного пользовательского интерфейса. 

Задачи, решаемые в ходе работы (в соответствии с заданием на НИР):
\begin{enumerate}
    \item Изучение метода «дефокусировки» и семантического анализа;
    \item Разработка концепции и архитектуры мобильных приложений, предназначенных для преобразования 2D в 3D;
    \item Анализ существующих продуктов со схожим функционалом;
    \item Подготовка отчета и презентации по НИР.
\end{enumerate}
