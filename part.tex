SERPENT [2] основан на методе Монте-Карло и может быть использован для
следующие цели: генерация однородных многогрупповых констант для детерминированных расчетов реакторов;
расчет нейтронно-физических характеристик для систем, содержащих делящийся материал; проверка детерминированных кодов;
нейтронно-физический расчет реактора и нуклидный состав ядерного топлива
определение.
SERPENT [2] позволяет создавать расчетные модели произвольных сложных систем, используя
в описании двух- или трехмерной геометрии. Способность SERPENT [2] к
моделировать сложную геометрию позволяет рассчитать «полноядерные» системы, т. е. моделировать полную загрузку реактора с возможностью описания каждого отдельного элемента, ячейки канала или активной зоны реактора (рис. 2).
Моделирование геометрии в SERPENT основано на описании отдельных элементов (а также в MCNP [12] и KENO [4]), что позволяет настраивать практически каждую двух- или трехмерную систему. Геометрия расчетной модели состоит из клеток, ограниченных элементарными поверхностями и их производными. Способность упрощенной установки конкретных геометрических объектов, таких как цилиндрические и сферические топливные элементы, квадратная и шестиугольная решетка и круговые кластеры из легкого водного реактора, характерные для CANDU и RBMK, также предоставлены в SERPENT.
В SERPENT используются библиотеки поперечных сечений взаимодействия нейтронных взаимодействий ACE-формата. В ходе расчетов можно использовать данные поперечного сечения таких универсально распознанных файлов, как ENDF/B, JEFF и т.д. Инсталляционный пакет включает библиотеки сечений для 432 нуклидов при 6 температурах (от 300 К до 1800 К) на основе JEF-2.2, JEFF-3.1, ENDF / B-VI.8, ENDF / B-VII. Инсталляционный пакет также включает библиотеки для термализации нейтронов для легкой воды, тяжелой воды и графита.
SERPENT позволяет размещать определяемые пользователем детекторы в любой точке системы для регистрации
различные интегральные параметры реакции [2]. Количество детекторов неограничено и их структура варьируется. Для расчетов доступны различные функции ответа.
Концентрации нуклидов, поперечные сечения трансмутации, данные о активности и распаде тепла для каждого материала и всей системы можно найти в выходных файлах расчета состава нуклидов [2]. Константы группы и другие выходные параметры вычисляются и записываются для каждого этапа выгорания. Все выходные файлы с численными значениями записываются в формате Matlab, что упрощает последующий анализ [2]. SERPENT позволяет создавать файлы с графическим представлением геометрии (включая визуализацию любых интенсивностей реакции), что удобно, особенно для визуализации распределения плотности нейтронного потока в системе
(Рисунок 3).





\section{Программный комплекс SERPENT2}
SERPENT — это универсальная трёхмерная непрерывная энергетическая  программа моделирования переноса частиц методом Монте-Карло. Был разработан в Техническом исследовательском центре VTT в Финляндии. Изначально, Serpent был в качестве упрощенной программой для физики реакторов, но возможности текущей версии разработки Serpent 2 значительно превосходят модели реактора. Приложения можно условно разделить на три категории:

\begin{enumerate}
	\item Традиционные приложения физики реакторов, включая пространственную гомогенизацию, расчеты критичности, исследования топливного цикла, моделирование исследовательских реакторов, проверку детерминированных транспортных кодов и т.д.;
	\item Мультифизические симуляции, т.е. расчеты, связанные с тепловой гидравликой, CFD и кодами производительности топлива;
	\item Моделирование переноса нейтронов и фотонов для расчета мощности дозы облучения, экранирования, исследований слияния и медицинской физики.
\end{enumerate}

Serpent также использует механизм параллельных вычислений, однако, необходимо принять во внимание, что многократные задачи совместно используют то же пространство памяти, следовательно, размер выделенной памяти также растет. И необходимо вручную регулировать емкость памяти при компиляции. Также стоит учесть, что данная методология еще не полностью протестирована.

В данной программе используются следующие типы данных: 
\begin{enumerate}
	\item Непрерывный нейтронный крест секции используются для фактического моделирования транспорта. Данные содержат все необходимые
	сечения реакции и все это вместе с энергетическими и угловыми распределениями дают выход нейтронов деления и задержанные параметры нейтронов.
	\item Cечение дозиметрии: дозиметрические сечения существуют
	для большого разнообразия материалов и могут включать в себя производные реакционные режимы не часто
	встречающиеся в транспортном расчете. Данные могут состоять из одного или нескольких частичных крестов, однако все энергетические и угловые распределения опущены. 
	\item Сечения термического рассеяния используются для замены упругого газа с низкой энергией реакции рассеяния для некоторых важных связанных замещающих нуклидов, таких как водород в воде или углерод в графите.
\end{enumerate}

Формат входных файлов не ограничен. Входной файл разделяется на отдельные блоки данных, обозначаемые как карты. Файл обработывается по одной карте за раз, и нет никаких ограничений на порядок, в котором должны быть организованы карточки.

Код Serpent использует универсальную модель геометрии для описания сложных структур, где геометрия разделена на отдельные уровни, которые все построены независимо и вложены друг в друга. Такой подход позволяет использовать регулярные геометрические структуры, такие как квадратные и шестиугольные решетки, обычно встречающиеся в реакторах. 

На самом высоком уровне, геометрия состоит из топливных штифтов, в которых топливные гранулы окружены оболочкой и охлаждающей жидкостью. Каждый тип штыря описывается независимо.
Следующий уровень представляет собой топливную сборку, в которой штыри расположены в правильной решетке. Сборка может также содержать стенки канала потока, каналы замедлителя или любую опору структур. На следующем уровне геометрии эти штыри сборки расположены в другой решетке для формирования основной компоновки, которая может быть окружена радиальными и осевыми отражателями и, наконец, стенку сосуда высокого давления реактора.

Основным строительным блоком геометрии является ячейка, которая определяется как область пространства, использующая простые граничные поверхности. Каждая ячейка заполнена однородным составом материала. 

\subsection{Геометрия}

Описание геометрии в коде Serpent состоит из двух- или трехмерных областей,
обозначенных как ячейки. Каждая ячейка определяется с использованием набора положительных и отрицательных чисел поверхности, которые соответствуют поверхностным идентификаторам, определенным в поверхностных карточках. В отличие от MCNP и другие кодов Монте-Карло, Serpent может обрабатывать только пересечения граничных поверхностей. Это означает, что нейтрон находится внутри ячейки, тогда и только тогда, когда он находится на одной стороне каждой границы
как указано в списке поверхностей.

Отсутствие оператора объединения ограничивает общность описания геометрии в некоторой степени. Это ограничение компенсируется за счет обеспечения большой коллекции типов производной поверхности, которые в большинстве случаев могут быть использованы для замены соединений элементарных поверхностей.
Преимущество такого подхода состоит в том, что описание геометрии остается относительно простым.

Топливный штырь не является фактическим объектом геометрии, а скорее макросом, который используется для определения пространств контактов. Области материала и их внешние радиусы приведены в восходящем, а код строит ячейки с использованием цилиндрических поверхностей. Если радиус отрицательный, он интерпретируется как толщина слоя вместо абсолютного радиуса. Номер пространства задается идентификатором штыря. Материалы контактов также могут быть другими пространствами, которые определяются с помощью команды заполнения. 
Определения контактов наиболее часто используются с решетками для определения топливных сборок.

Топливные штыри и частицы являются частным случаем геометрии "гнезда". Гнезда могут
быть определены с помощью планарных (px, py, pz), цилиндрических (цил, sqc, hexxc, hexyc), сферических (sph) или кубических (кубических) типов поверхности. 

\subsection{Материал}

Каждый материал состоит из списка нуклидов, и каждый нуклид связан с библиотекой поперечного сечения, как определено в файле. Температуры нуклида фиксируются, когда данные поперечного сечения генерируются и не могут быть изменены впоследствии. Важно использовать библиотеки поперечного сечения, созданные при правильной температуре, чтобы правильно моделировать допплеровское уширение резонансных пиков. В равной степени (или даже более) важно использовать соответствующие библиотеки термического рассеяния связанного атома для замедлителей нуклидов.

Имена нуклидов могут быть произвольными псевдонимами, определенными в файле каталога. Важным моментом является то, что названия нуклидов используются только для идентификации, и они не содержат никакой информации, используемой кодом в расчетах.

Имя материала используется для идентификации материала на карточках. Названия нуклидов соответствуют идентификатору, определенному в файле каталога. Эти идентификаторы определяют данные поперечного сечения, используемые в расчете. Плотности и фракции могут быть заданы как атомные или массовые значения. Положительные данные относятся к атомным плотностям и атомным фракциям соответственно, а отрицательные значения относятся к массовым плотностям и массовым фракциям. Объемы материалов и массы используются для нормализации скоростей реакции, что важно, например, при расчете выгорания. Код вычисляет их автоматически для простых структур контактов, но для некоторых более сложных геометрий значения должны вводиться вручную. 

Режим расчета по умолчанию в Serpent - это метод источника критичности k-собственного значения, в котором симуляция выполняется циклически, а распределение источника каждого цикла формируется распределение реакции деления предыдущего цикла.

Число исходных нейтронов за цикл фиксировано. Так как количество сгенерированных исходных точек обычно отличается от этого значения, размер источника увеличивается (keff <1) или уменьшается (keff> 1), чтобы соответствовать заданному размеру источника. Нейтральные циклы - это циклы, которые запускаются, чтобы позволить исходному распределению исходного деления сходиться, прежде чем начинать собирать результаты. В решеточных расчетах сходимость обычно достигается хорошо в течение первых 20 циклов. Однако конвергенция источников в полномасштабных расчетах может занять гораздо больше времени. 

\subsection{Детекторы}

рпоп