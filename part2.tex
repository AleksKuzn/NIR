% второй раздел - файл part2.tex

\section{Обработка результатов расчетов ПК SERPENT2}

Главной задачей на текущий семестр является создание скриптов, которые смогут помочь автоматизировать функции, необходимые для определения нуклидного состава ядерного топлива, который будет использоваться в качестве входных условий для SERPENT2. 

\subsection{Python-скрипт для поиска и извлечения информации из расчетных файлов}

После того, как была проанализорована структура расчетных файлов, можно начинать приступать к извлечению и анализу информации, которая в них содержится.

Для этого необходимо реализовать скрипты, которые могли бы:

\begin{enumerate}
	\item Осуществлять поиск конкретных видов топлива, в текстовом файле начальных условий.
	\item Получать данные о концентрации нуклидов конкретного топлива в выходном файле расчета выгорания.
\end{enumerate}

Для решения поставленных задач были реализованы скрипты на языке Python \cite{python}:

\begin{enumerate}
	\item Searher.py
	\item fuelGetter.py
\end{enumerate}

Класс Searcher.py содержит функцию, необходимые для поиска содержащихся видов топлива (fuel) в текстовом файле начальных условий (листинг ~\ref{fuelList}), а также  функцию, которая извлекает для каждого вида топлива данные об изменении концетрации нуклидов топлива во времени из файла расчета выгорания(листинг ~\ref{adens}). 

%Для отображения исходных кодов (см. пример ссылки на литературу \cite{Rrus1}) можно  воспользоваться окружением \verb|MyCode|:

%\begin{MyCode}
%require(ggplot2)
%gi = ggplot(data = iris, aes(x=Sepal.Length,y=Petal.Length))
%gi + geom_point(aes(color=Species))
%\end{MyCode}

%Для нумерации листингов и возможности ссылаться на них  предназначено окружение \verb|Program|.

\begin{Program}[H]
\begin{MyCode}
@staticmethod
def search_in_txt_file(filename):
"""метод ищет в строке, если есть burn соответствующий fuel и
формирует массив для посика соответствующего ADENS"""
f = open(filename, 'r')
l_burns = list() #создаем список строк для fuel - формата:
					 MAT_fuel ... _ADENS
subs = "fuel"
substr = "burn"
pattern_part1 = "MAT_"   # was 'MAT_fuel'
pattern_part2 = "_ADENS = ["
for line in f:  # для каждой строки из файла ищем
	if substr in line:  # если строка содержит слово burn
		s = line.split()  # разбиваем ее по пробелам
		for ss in s:  #для каждого элемента 
		разбиенной строки ещем
			if subs in ss: #содержит ли элемент
				подстроку fuel
			   hvost = ss[ss.rfind(subs): len(ss)]  
				#создаем переменную, которая
				 будет содержать 
				соответствующее число,
				 стоящее после fuel
		l_burns.append(pattern_part1 + hvost +
						 pattern_part2) 
		  # формируем строку формата: 
			 	MAT_fuel+hvost_ADENS

return l_burns
\end{MyCode}
\label{fuelList}
\caption{Функция получения списка видов топлива}
\end{Program}

\begin{Program}[H]
\begin{MyCode}
		@staticmethod
		def search_in_m_file(l_burns, filesPath, m_filename):
		"""Метод ищет в .m файле ADENS-ы и извлекает все, что
		находится между квадратными скобками, для каждого fuel"""
		last_symbol = "]"
		m_file = open(m_filename, 'r') #открываем .m файл для чтения 
		и нахождения соответствующих ADENS
		text = m_file.read()
		for l in l_burns:  #для каждого элемента массива, 
		содержащего паттерны ADENS
		if l in text:
		start = text.find(l)+ len(l)
		end = text.find(last_symbol, start)
		adens = text[start : end]
		filename = str(filesPath + l + ".txt")
		adens_file = open(filename, "w")
		adens_file.write(adens)
		adens_file.close()		
\end{MyCode}
\label{adens}
\caption{Функция извлечения данных об изменении концетрации нуклидов топлива во времени}
\end{Program}

А скрипт fuelGetter.py запускает фукнцию получения списка видов топлива и будет использоваться в Java-приложении.


\subsection{Реалиация Java-приложения}

\subsubsection{Стартовое окно}

Теперь, когда скрипты для извлечения данных о топливе и его видах получены, можно реализовать Java-приложение \cite{shild} для управления обработкой результатов расчетов SERPENT-а. 

При запуске приложения появляется следующее окно (рисунок ~\ref{fig:startWindow}):

\begin{figure}[H]
	\centering
	\includegraphics[width=0.5\linewidth]{pics/startWindow}
	\caption{Стартовое окно приложения}
	\label{fig:startWindow}
\end{figure}

Данное окно содержит активную кнопку, по нажатию которой можно получить список всех видов топлива из текстового файла. Иными словами, она запускает Python-скрипт fuelGetter.py и результаты его работы выводятся в окно Java-приложения.

Как только список видов топлива был получен, можно начать работать с конкретным видом. После выбора определенного вида топлива становятся активными еще 2 кнопки(рисунок ~\ref{fig:startActive}): "Получить информацию о виде топлива" и "Информация о концентрации вида топлива".

\begin{figure}[H]
	\centering
	\includegraphics[width=0.5\linewidth]{pics/startActive}
	\caption{Начало работы с конкретным видом топлива}
	\label{fig:startActive}
\end{figure}

\subsubsection{Получение информации о виде топлива}

При нажатии на кнопку  "Получить информацию о виде топлива" пользователь получает полный список нуклидов, из которых состоит выбранный вид топлива:

\begin{figure}[H]
	\centering
	\includegraphics[width=0.5\linewidth]{pics/nuclides}
	\caption{Начало работы с конкретным видом топлива}
	\label{fig:nuclides}
\end{figure}

По каждому нуклиду можно будет просмотреть изменение его концентраций, при нажатии соответствующей кнопки (рисунки ~\ref{fig:nuclideList}, ~\ref{fig:nuclideList2}):

\begin{figure}[H]
	\centering
	\includegraphics[width=0.5\linewidth]{pics/nuclideList}
	\caption{Начало работы с конкретным видом топлива}
	\label{fig:nuclideList}
\end{figure}

\begin{figure}[H]
	\centering
	\includegraphics[width=0.5\linewidth]{pics/nuclideList2}
	\caption{Начало работы с конкретным видом топлива}
	\label{fig:nuclideList2}
\end{figure}

Также, выбранный нуклид можно удалить из списка нуклидов, составляющих топливо.

\subsubsection{Получение информации о концентрации вида топлива} 

Если же пользователю хочется увидеть изменение концентраций всех нуклидов и выбрать определенный столбец концентраций в качестве начальных условий для данного вида топлива, то в стартовом окне ему необходимо нажать на кнопку "Информация о концентрации вида топлива". Необходимая информация появится в таблице (рисунок  ~\ref{fig:table}). 

\begin{figure}[H]
	\centering
	\includegraphics[width=0.9\linewidth]{pics/table}
	\caption{Изменение концентраций всех нуклидов во времени}
	\label{fig:table}
\end{figure}

Любой из столбцов таблицы может быть выбран в качестве начальных условий для данного вида топлива(начальные условия описываются в текстовом файле). Для этого пользователь может нажать на кнопку <<Сделать начальным условием>> и указанный пользователем столбец будет запишется в новый текстовый файл начальных условий как показано на рисунке ниже:

\begin{figure}[H]
	\centering
	\includegraphics[width=0.5\linewidth]{pics/newtxt}
	\caption{Текстовый файл начальных условий}
	\label{fig:newtxt}
\end{figure}

Эти же выбранные начальные условия можно отобразить на графике с использованием логарифмической шкалы. Для этого необходимо перейти по кнопке <<Изобразить>>. 

\begin{figure}[H]
	\centering
	\includegraphics[width=0.5\linewidth]{pics/diagram}
	\caption{Пример динамики концентрации радионуклида}
	\label{fig:diagram}
\end{figure}

%Следует учесть, что объем кода в таком листинге должен быть небольшим --- меньше страницы, а нумерация появляется, если внутри окружения задано название листинга с помощью команды \verb|\caption|. 

%Кроме того, положение кода выбирается системой \LaTeXe.